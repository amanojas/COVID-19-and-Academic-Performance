% Options for packages loaded elsewhere
% Options for packages loaded elsewhere
\PassOptionsToPackage{unicode}{hyperref}
\PassOptionsToPackage{hyphens}{url}
\PassOptionsToPackage{dvipsnames,svgnames,x11names}{xcolor}
%
\documentclass[
  12pt,
  letterpaper,
  DIV=11,
  numbers=noendperiod]{scrartcl}
\usepackage{xcolor}
\usepackage[margin=1in]{geometry}
\usepackage{amsmath,amssymb}
\setcounter{secnumdepth}{-\maxdimen} % remove section numbering
\usepackage{iftex}
\ifPDFTeX
  \usepackage[T1]{fontenc}
  \usepackage[utf8]{inputenc}
  \usepackage{textcomp} % provide euro and other symbols
\else % if luatex or xetex
  \usepackage{unicode-math} % this also loads fontspec
  \defaultfontfeatures{Scale=MatchLowercase}
  \defaultfontfeatures[\rmfamily]{Ligatures=TeX,Scale=1}
\fi
\usepackage{lmodern}
\ifPDFTeX\else
  % xetex/luatex font selection
  \setmainfont[]{Palatino}
  \setsansfont[]{Palatino}
  \setmonofont[]{Courier New}
  \setmathfont[]{Palatino}
\fi
% Use upquote if available, for straight quotes in verbatim environments
\IfFileExists{upquote.sty}{\usepackage{upquote}}{}
\IfFileExists{microtype.sty}{% use microtype if available
  \usepackage[]{microtype}
  \UseMicrotypeSet[protrusion]{basicmath} % disable protrusion for tt fonts
}{}
\usepackage{setspace}
\makeatletter
\@ifundefined{KOMAClassName}{% if non-KOMA class
  \IfFileExists{parskip.sty}{%
    \usepackage{parskip}
  }{% else
    \setlength{\parindent}{0pt}
    \setlength{\parskip}{6pt plus 2pt minus 1pt}}
}{% if KOMA class
  \KOMAoptions{parskip=half}}
\makeatother
% Make \paragraph and \subparagraph free-standing
\makeatletter
\ifx\paragraph\undefined\else
  \let\oldparagraph\paragraph
  \renewcommand{\paragraph}{
    \@ifstar
      \xxxParagraphStar
      \xxxParagraphNoStar
  }
  \newcommand{\xxxParagraphStar}[1]{\oldparagraph*{#1}\mbox{}}
  \newcommand{\xxxParagraphNoStar}[1]{\oldparagraph{#1}\mbox{}}
\fi
\ifx\subparagraph\undefined\else
  \let\oldsubparagraph\subparagraph
  \renewcommand{\subparagraph}{
    \@ifstar
      \xxxSubParagraphStar
      \xxxSubParagraphNoStar
  }
  \newcommand{\xxxSubParagraphStar}[1]{\oldsubparagraph*{#1}\mbox{}}
  \newcommand{\xxxSubParagraphNoStar}[1]{\oldsubparagraph{#1}\mbox{}}
\fi
\makeatother


\usepackage{longtable,booktabs,array}
\usepackage{calc} % for calculating minipage widths
% Correct order of tables after \paragraph or \subparagraph
\usepackage{etoolbox}
\makeatletter
\patchcmd\longtable{\par}{\if@noskipsec\mbox{}\fi\par}{}{}
\makeatother
% Allow footnotes in longtable head/foot
\IfFileExists{footnotehyper.sty}{\usepackage{footnotehyper}}{\usepackage{footnote}}
\makesavenoteenv{longtable}
\usepackage{graphicx}
\makeatletter
\newsavebox\pandoc@box
\newcommand*\pandocbounded[1]{% scales image to fit in text height/width
  \sbox\pandoc@box{#1}%
  \Gscale@div\@tempa{\textheight}{\dimexpr\ht\pandoc@box+\dp\pandoc@box\relax}%
  \Gscale@div\@tempb{\linewidth}{\wd\pandoc@box}%
  \ifdim\@tempb\p@<\@tempa\p@\let\@tempa\@tempb\fi% select the smaller of both
  \ifdim\@tempa\p@<\p@\scalebox{\@tempa}{\usebox\pandoc@box}%
  \else\usebox{\pandoc@box}%
  \fi%
}
% Set default figure placement to htbp
\def\fps@figure{htbp}
\makeatother


% definitions for citeproc citations
\NewDocumentCommand\citeproctext{}{}
\NewDocumentCommand\citeproc{mm}{%
  \begingroup\def\citeproctext{#2}\cite{#1}\endgroup}
\makeatletter
 % allow citations to break across lines
 \let\@cite@ofmt\@firstofone
 % avoid brackets around text for \cite:
 \def\@biblabel#1{}
 \def\@cite#1#2{{#1\if@tempswa , #2\fi}}
\makeatother
\newlength{\cslhangindent}
\setlength{\cslhangindent}{1.5em}
\newlength{\csllabelwidth}
\setlength{\csllabelwidth}{3em}
\newenvironment{CSLReferences}[2] % #1 hanging-indent, #2 entry-spacing
 {\begin{list}{}{%
  \setlength{\itemindent}{0pt}
  \setlength{\leftmargin}{0pt}
  \setlength{\parsep}{0pt}
  % turn on hanging indent if param 1 is 1
  \ifodd #1
   \setlength{\leftmargin}{\cslhangindent}
   \setlength{\itemindent}{-1\cslhangindent}
  \fi
  % set entry spacing
  \setlength{\itemsep}{#2\baselineskip}}}
 {\end{list}}
\usepackage{calc}
\newcommand{\CSLBlock}[1]{\hfill\break\parbox[t]{\linewidth}{\strut\ignorespaces#1\strut}}
\newcommand{\CSLLeftMargin}[1]{\parbox[t]{\csllabelwidth}{\strut#1\strut}}
\newcommand{\CSLRightInline}[1]{\parbox[t]{\linewidth - \csllabelwidth}{\strut#1\strut}}
\newcommand{\CSLIndent}[1]{\hspace{\cslhangindent}#1}



\setlength{\emergencystretch}{3em} % prevent overfull lines

\providecommand{\tightlist}{%
  \setlength{\itemsep}{0pt}\setlength{\parskip}{0pt}}



 


\usepackage{booktabs}
\usepackage{longtable}
\usepackage{array}
\usepackage{multirow}
\usepackage{wrapfig}
\usepackage{float}
\usepackage{colortbl}
\usepackage{pdflscape}
\usepackage{tabu}
\usepackage{threeparttable}
\usepackage{threeparttablex}
\usepackage[normalem]{ulem}
\usepackage{makecell}
\usepackage{xcolor}
\KOMAoption{captions}{tableheading}
\usepackage{makecell}
\usepackage{booktabs}
\usepackage{endfloat}
\usepackage{threeparttable}
\usepackage{siunitx}
\sisetup{detect-all=true}
\setlength{\defaultaddspace}{0pt}
\providecommand\makecell[2][]{\begin{tabular}{@{}c@{}}#2\end{tabular}}
\makeatletter
\@ifpackageloaded{caption}{}{\usepackage{caption}}
\AtBeginDocument{%
\ifdefined\contentsname
  \renewcommand*\contentsname{Table of contents}
\else
  \newcommand\contentsname{Table of contents}
\fi
\ifdefined\listfigurename
  \renewcommand*\listfigurename{List of Figures}
\else
  \newcommand\listfigurename{List of Figures}
\fi
\ifdefined\listtablename
  \renewcommand*\listtablename{List of Tables}
\else
  \newcommand\listtablename{List of Tables}
\fi
\ifdefined\figurename
  \renewcommand*\figurename{Figure}
\else
  \newcommand\figurename{Figure}
\fi
\ifdefined\tablename
  \renewcommand*\tablename{Table}
\else
  \newcommand\tablename{Table}
\fi
}
\@ifpackageloaded{float}{}{\usepackage{float}}
\floatstyle{ruled}
\@ifundefined{c@chapter}{\newfloat{codelisting}{h}{lop}}{\newfloat{codelisting}{h}{lop}[chapter]}
\floatname{codelisting}{Listing}
\newcommand*\listoflistings{\listof{codelisting}{List of Listings}}
\makeatother
\makeatletter
\makeatother
\makeatletter
\@ifpackageloaded{caption}{}{\usepackage{caption}}
\@ifpackageloaded{subcaption}{}{\usepackage{subcaption}}
\makeatother
\usepackage{bookmark}
\IfFileExists{xurl.sty}{\usepackage{xurl}}{} % add URL line breaks if available
\urlstyle{same}
\hypersetup{
  pdftitle={Effects of COVID-19 on the Academic Performance of College Students},
  pdfauthor={Aman Desai},
  colorlinks=true,
  linkcolor={blue},
  filecolor={Maroon},
  citecolor={Blue},
  urlcolor={blue},
  pdfcreator={LaTeX via pandoc}}


\title{Effects of COVID-19 on the Academic Performance of College
Students}
\author{Aman Desai}
\date{Feb 2, 2026}
\begin{document}
\maketitle
\begin{abstract}
I analyze the impact of the COVID-19 pandemic on undergraduates'
performance in an introductory economics course at a large public
university. One challenge in analyzing student academic outcomes during
the pandemic was the explicit change in grading policies by college
administrators as well as the implicit adjustment by faculty designed to
mitigate the impact of an abrupt shift to online learning amidst the
stress and uncertainty associated with the pandemic. To limit the impact
of grading policies, I analyze changes in the raw scores on a common
final administered to all sections of the course in the year before and
for four semesters after Spring 2020. To limit variation in the
difficulty of the exams from before to during the pandemic, I compare
student performance on nearly identical questions on the final exam over
time. Adjusted mean scores on the common final fell by a point and the
probability of answering the qualitatively same question on the final
fell, on average, by 1.5 percentage points. Students with lower GPAs
were 3.3 percentage points (or 0.02 standard deviations) less likely to
answer similar questions correctly relative to students with higher GPAs
during the pandemic. Also, the mean probability of answering a nearly
identical question before and after suddenly moving to online classes
increased by 5.6 percentage points.

\vspace{12pt}

\noindent \textbf{Keywords:} COVID-19, academic performance,
undergraduate education, online learning
\end{abstract}


\setstretch{2}
\pagebreak

\section{Introduction}\label{introduction}

The COVID-19 pandemic of March 2020 was disruptive across many domains,
with higher education being one of them. Policies were implemented
worldwide in response to this global crisis, resulting in changes in the
educational setting. Educational instructions were abruptly moved online
without prior preparation. This had a negative effect on primary and
secondary education, leading to significant learning loss for students
(\citeproc{ref-grewenig_covid-19_2021}{Grewenig et al. 2021};
\citeproc{ref-fuchs-schundeln_covid-induced_2022}{Fuchs-Schündeln
2022}).

Although the socioeconomic consequences of COVID-19 have been
extensively studied from various perspectives, research on the impact of
the pandemic on college students remains limited and yields conflicting
results. Most studies examining the impact of the pandemic on students'
academic performance measure outcomes such as GPA and course completion
rates. Although useful, these measures are confounded by the numerous
responses of students, faculty, and administrators during the pandemic.
For example, cheating became a challenging issue in the rapid move to
online teaching (\citeproc{ref-ives_did_2024}{Ives and Cazan 2024};
\citeproc{ref-jenkins_when_2023}{Jenkins et al. 2023};
\citeproc{ref-walsh_why_2021}{Walsh et al. 2021}). The faculty adopted
more lenient grading practices and reduced exam difficulties.
Administrators altered grading policies regarding course withdrawals and
pass/fail options
(\citeproc{ref-rodriguez-planas_covid-19_2022}{Rodríguez-Planas 2022}).
These responses made comparisons with pre-pandemic test scores, or the
term GPA less reliable for quantifying learning loss during the pandemic
in college students.

I address these assessment and grading issues using unique exam-level
data from a large public university in New York City. First, I analyze
students' performance on final exams before and during the pandemic in
an \emph{introductory microeconomics} course. Approximately 800 students
across ten sections of the course attempted a common final exam each
semester. This reduces the variation in the difficulty of exams across
sections. However, the difficulty of an exam may have changed in
response to the pandemic. Thus, I compare students' performance on
specific exam questions that were qualitatively almost identical before
and during the pandemic by matching questions from answer sheets from
the common final exams before and during the pandemic. By focusing on
students' performance on nearly identical questions before and during
the pandemic, I remove the variation in outcomes due to possible changes
in the difficulty of these exams during the pandemic. By combining the
matched question-level data with student characteristics, I estimate how
the pandemic affected students' average probability of correctly
answering similar questions from pre-pandemic common exams during the
crisis.

I begin with a before and after analysis, adjusting for student
characteristics, time periods, and instructor fixed effects. I argue
that more capable students are more likely to adjust to online
instruction more effectively. Thus, I use a difference-in-difference
design and compared students with pre-course GPA above (high GPA) and
below (low GPA) the median before and during the pandemic. I observe
that during the pandemic, low-GPA students were less likely to answer
qualitatively similar questions from the pre-pandemic exams relative to
students with higher GPAs. My analysis of dynamic effects reveals that
by Spring 2022, the performance gap persisted between low and high GPA
students, both in overall exam scores and in their likelihood of
correctly answering nearly identical questions compared with
pre-pandemic levels.

I also analyze the students' performance on matched questions by
difficulty level. I find no statistically significant impact of the
pandemic on low GPA students' average probability of answering nearly
identical ``easy'' questions correctly, but I find significant effect on
their performance with nearly identical \emph{hard} questions. I also
provide a similar analysis comparing outcomes between students enrolled
in online and hybrid classes. My findings align with existing research
on learning loss during this period. By analyzing the performance of
qualitatively similar exam questions before and during the pandemic, I
contribute to the literature by offering more reliable estimates of
learning loss compared to traditional metrics, such as GPA and course
withdrawals.

The next section reviews the current literature on the effects of the
pandemic on college students' academic outcomes. Section 3 discusses the
data, section 4 explains the estimation strategy, section 5 reports the
results, and section 6 concludes the paper.

\section{Literature Review}\label{literature-review}

Most early studies analyzing the impact of COVID-19 on undergraduate
student outcomes were based on surveys about their experiences during
the pandemic. Jaeger et al. (\citeproc{ref-jaeger_global_2021}{2021})
was the first to document the negative impact of the COVID-19 pandemic
using surveys administered to university students in 28 universities in
the United States, Spain, Australia, Sweden, Austria, Italy, and Mexico
between April and October 2020. Their preliminary results reported
disparate impacts on different socio-economic and demographic groups.
Aucejo et al. (\citeproc{ref-aucejo_impact_2020}{2020}), one of the
first papers studying the effect of COVID-19 on college student
outcomes, surveyed 1,500 students at a large public university in the
United States. They found significant negative effects of the pandemic
on student outcomes. Due to the pandemic, 13\% of students delayed
graduation, 40\% lost a job, internship, or offer, and 29\% expected an
earnings loss by age 35. They also found large disparate impacts of the
pandemic across socio-economic statuses. Lower-income students were 55\%
more likely than their higher-income peers to have delayed graduation
due to COVID-19.

Along the same lines, Rodríguez-Planas
(\citeproc{ref-rodriguez-planas_hitting_2020}{2020}) collected data on
students' experiences during the pandemic using an online survey at an
urban public college in New York City in the summer of 2020. The author
found significant disruptions in students' lives due to the pandemic.
Because of COVID, between 14\% and 34\% of students considered dropping
a class during Spring 2020, 30\% modified their graduation plans, and
the freshman Fall retention rate dropped by 26\%. The pandemic also
deprived 39\% of students of their jobs, while 35\% of students saw
their earnings reduced. Pell grant recipients (students from
lower-income families) were 20\% more likely to lose a job due to the
pandemic and 17\% more likely to experience earning losses than non-Pell
recipients. Other vulnerable groups, such as first-generation and
transfer students, were relatively more affected. Since they seem to
rely less on financial aid and more on income from wage and salary jobs,
both their educational and employment outcomes were more negatively
impacted by the pandemic compared to students whose parents also
attended college or those who began college as freshmen.

The pandemic's impact on student learning was largely driven by the
sudden shift to remote instruction. Literature on remote learning shows
various approaches including fully remote, software-assisted, and hybrid
learning\footnote{see Escueta et al.
  (\citeproc{ref-escueta_education_2017}{2017}) for a comprehensive
  review.}. While online learning offers reduced costs in delivering
education and wider accessibility, research indicates mixed results.
Studies using randomized trials found that students in remote formats
generally performed worse than those in traditional settings Alpert,
Couch, and Harmon (\citeproc{ref-alpert_randomized_2016}{2016}).
Bettinger et al. (\citeproc{ref-bettinger_virtual_2017}{2017}) and
Cacault et al. (\citeproc{ref-cacault_distance_2021}{2021}) found that
online learning particularly disadvantaged lower-performing students.
Multiple analyses have demonstrated that online courses lead to lower
completion rates, grades, and persistence
(\citeproc{ref-jaggars_how_2016}{Jaggars and Xu 2016};
\citeproc{ref-xu_promises_2019}{Xu and Xu 2019}).

Several studies attempt to use the pandemic as an exogenous shock to
measure the impact of remote learning on college students' outcomes. For
instance, in their study, Altindag, Filiz, and Tekin
(\citeproc{ref-altindag_is_2021}{2021}) analyzed administrative data
from a public university and employed a fixed effects model. They
examine the effect of the change in learning modality due to the
pandemic on students' learning outcomes. They found that the online
instruction mode led to lower grades and an increased likelihood of
course withdrawal. Students who have had greater exposure to in-person
instruction have a lower likelihood of course repetition, a higher
probability of graduating on time, and achieving a higher graduation
GPA. Additionally, they observed that the difference in student
performance between in-person and online courses tended to diminish over
time in the post-pandemic era.

In the fall of 2020, Kofoed et al.
(\citeproc{ref-kofoed_zooming_2021}{2021}) randomized 551 West Point
students in a required introductory economics course across twelve
instructors into either an online or in-person class. They found that
final grades for online students dropped by 0.215 standard deviations.
This result was apparent in both assignments and exams and was largest
for academically at-risk students. Additionally, using a post-course
survey, they found that online students struggled to concentrate in
class and felt less connected to their instructors and peers. They
conclude that the shift to online education had negative effects on
learning. Using data on Virginia community college students, Bird,
Castleman, and Lohner (\citeproc{ref-bird_negative_2022}{2022}) applied
a difference-in-differences research design leveraging instructor fixed
effects and student fixed effects to estimate the impact of the
transition to online learning due to the pandemic. Their results show a
modest negative impact of 3\% - 6\% on course completion. Additionally,
their findings suggest that faculty experience in delivering online
lectures does not mitigate the negative effects. In their exploratory
analyses, they find minimal long-term effects of the switch to online
learning.

A comprehensive study by Bonacini, Gallo, and Patriarca
(\citeproc{ref-bonacini_unraveling_2023}{2023}), disentangle the
channels through which the pandemic affected students. They use admin
data from 2018-2021 of 36,000 university students in Italy who took
about 400,000 exams during this period. They examine the overall effect
of the pandemic on students' exam scores in different courses.
Additionally, they explore the effect of the transition to remote
learning by using COVID as an exogenous shock with a
difference-in-differences design. Their findings show that during the
pandemic, students performed better, with an increase in exam scores.
However, the abrupt move to remote learning decreased students' exam
scores.

Studies using survey data on students discussed above have found a
negative impact of COVID-related disruptions on academic performance.
However, studies that use measured outcomes to evaluate academic
performance report mixed results, especially immediately after the
pandemic began (\citeproc{ref-bird_negative_2022}{Bird, Castleman, and
Lohner 2022}; \citeproc{ref-bonacini_unraveling_2023}{Bonacini, Gallo,
and Patriarca 2023}). One reason for this might be that many
institutions temporarily implemented policies to reduce the burden on
students during the pandemic, particularly due to the sudden transition
from traditional to fully remote learning. Instructors were likely more
lenient in setting exam questions and grading, and more willing to
accommodate students than before the pandemic. The sudden move to remote
learning could have also created more opportunities for misbehavior by
students during exams. For instance, Rodríguez-Planas
(\citeproc{ref-rodriguez-planas_covid-19_2022}{2022}), using data from
an urban public college in NYC found that lower-income students were 35
percent more likely to utilize the flexible pass/fail grading policy.
While no GPA advantage is observed among top-performing lower-income
students, in the absence of the flexible grading policy these students
would have seen their GPA decrease by 5\% relative to their pre-pandemic
mean.

The literature has provided valuable insights into the impact of the
COVID-19 pandemic on undergraduates. However, several issues remain to
be addressed. Many studies rely on self-reported survey data, which may
not accurately capture the true extent of learning loss
(\citeproc{ref-aucejo_impact_2020}{Aucejo et al. 2020};
\citeproc{ref-rodriguez-planas_hitting_2020}{Rodríguez-Planas 2020}). I
identify major limitations in these recent studies. First, using course
completion rates, course GPAs, or end-of-semester GPAs to measure
academic outcomes immediately after COVID-19 hit in March may not
accurately reflect students' actual learning or learning loss. Second,
the pandemic-driven sudden transition to new instruction modalities
likely changed assessment methods as instructors and students took time
to adjust to the situation. The difficulty of exams immediately after
the adjustment may not have been the same as pre-COVID exams,
contributing to inaccurate measurement of learning loss. Additionally,
the implementation of flexible grading policies may have biased the
effect of the pandemic on course GPA or course completion rates. I
contribute to the literature in two ways. To address these limitations,
I analyze students' performance on common exams before and during the
pandemic. To eliminate variation due to changes in the difficulty of
exams during the pandemic, I examine students' performance on nearly
identical questions from exams before and during the pandemic to measure
learning loss.

\section{Data}\label{data}

The data for this study are drawn from two primary sources covering the
academic years 2019 through 2022. The first source records student
performance on common departmental final examinations for the
\emph{Introductory Microeconomics} course at a large public university
in New York City. This course is offered every semester, taught by
multiple instructors, and at least 700 students enroll annually.

The department offers the course through three modalities. \emph{Hybrid}
sections meet twice weekly, comprising one in-person session and one
fully remote session. \emph{Online} sections are conducted entirely
remotely using software. In 2019 (Spring and Fall), the course was
primarily offered in the hybrid mode, with one large online section.
Following the onset of the COVID-19 pandemic---spanning Fall 2020
through Spring 2022---offerings were exclusively online or hybrid, with
the exception of a single in-person section in 2022. I do not include
those students in the analyses to facilitate the comparison between the
efficacy of hybrid and online learning modes.

Although the course involves multiple instructors and teaching
modalities, assessment is standardized; all enrolled students are
required to complete a common, multiple-choice final examination with a
maximum score of 40 points. Leveraging performance on common exams
eliminates potential bias arising from heterogeneity in
instructor-specific testing difficulty. The dataset includes answer
sheets for all students who attempted these exams, providing the final
score, item-level performance, instructor identifiers, and the course
learning mode. Data for the Spring 2020 semester were unavailable.

The first measure is the aggregate score on the common final
examination, converted from a raw maximum of 40 points to a standard
0--100 scale. This standardized measure provides a more consistent
signal of learning than course GPA or withdrawal rates, which were
potentially confounded by flexible grading policies adopted during the
pandemic.

To exploit the granularity of the data, I also construct a second
measure based on item-level performance. Because the final exams are
departmental, I am able to match identical or nearly identical questions
appearing in exams administered both pre- and post-pandemic onset. While
the department utilizes two versions of the exam to deter cheating
(differing only in question order), the content remains constant. I
manually identified and matched 35 unique pairs of questions across the
pre-pandemic and pandemic periods. This allows for the construction of a
binary outcome variable, \emph{correct}, which takes a value of 1 if a
student answered the matched question correctly and 0 otherwise.

The second data source consists of institutional administrative records
for all students enrolled in \emph{Introductory Microeconomics} during
the relevant semesters. This dataset includes a rich set of covariates,
including gender, race, age, transfer status, enrollment intensity
(part-time vs.~full-time), native language, and class standing (freshman
through senior).

By merging these administrative records with the examination data, I
construct a comprehensive dataset linking student characteristics to
standardized performance metrics. This merger also incorporates
exam-level characteristics, such as learning modality, course
instructor, exam version, and the semester of administration. To my
knowledge, this is the first dataset to facilitate a granular
examination of COVID-19's impact on student performance using
standardized, item-level outcomes.

The final analytical sample comprises 4,655 unique students enrolled in
the course. For the granular analysis using matched exam questions, the
dataset expands to 47,589 student-question observations. Each
observation represents a student-question pair indicating whether the
specific item was answered correctly. For the majority of the sample, I
utilize the cumulative GPA recorded prior to the start of the semester.
If the pre-semester GPA is unavailable, I substitute it with the GPA
calculated at the end of the concurrent semester. In cases where both
values are missing, I impute the missing value using the mean GPA of the
student cohort for that specific semester.

\section{Estimation Strategy}\label{estimation-strategy}

To estimate the impact of the pandemic on student learning, I employ a
series of Ordinary Least Squares regressions. I first examine aggregate
performance on the common final examination, followed by an analysis of
item-level performance using matched questions across pre-pandemic and
pandemic semesters. My primary empirical strategy estimates the effect
of the pandemic on student outcomes using the following specification:

\begin{equation}
y_{i,c,t} = \delta P_{t} + \beta X_{i,c,t} + \gamma_{c} + \alpha_{s} + \epsilon_{i,c,t}
\end{equation}

\(y_{i,c,t}\) represents the academic outcome for student \(i\) in class
taught by instructor \(c\) during semester \(t\). I analyze two distinct
outcomes for \(y\). In the first set of regressions, \(y\) is the
students' aggregate score on the common final examination, scaled to a
0--100 range. In the second set of regressions, I employ a linear
probability model where \(y\) is a binary indicator equal to 1 if the
student answered a specific matched question correctly, and 0 otherwise.
For this item-level analysis, the subscript \(q\) is added to denote the
specific question pair, such that the outcome is denoted as
\(y_{i,c,q,t}\).

The variable of interest is \(P_{t}\), a binary indicator for the
pandemic period, equal to 1 for any semester after Fall 2019 (i.e.,
Spring 2020 onward) and 0 otherwise. The coefficient \(\delta\) captures
the average effect of the pandemic on student performance.

\(X_{i,c,t}\) includes student-level controls to account for demographic
and academic heterogeneity. These include dummy variables for race
(Black, Asian, non-White Hispanic, and Other, with White as the
reference group), gender (equal to 1 if female), and class standing
(equal to 1 if the student is a freshman or sophomore). To control for
baseline ability, I include the student's cumulative GPA prior to the
start of the course. The model also includes instructor fixed effects
(\(\gamma_{c}\)) and session fixed effects (\(\alpha_{s}\)) to control
for time-invariant instructor characteristics and semester-specific
shocks unrelated to the pandemic. Standard errors are robust to
heteroskedasticity.

\subsection{Identification of Differential Impact of COVID-19 on Low vs
High GPA
Students}\label{identification-of-differential-impact-of-covid-19-on-low-vs-high-gpa-students}

I further examine whether the pandemic differentially affected students
based on their prior academic performance. I classify students into
``Low GPA'' and ``High GPA'' groups based on the sample median
cumulative GPA of 3.32. Students below this threshold are defined as
low-GPA, while those at or above are defined as high-GPA. I estimate the
following interaction model:

\begin{equation}
y_{i,c,t} = \delta P_{t} + \phi L_{i} + \mu P_{t} \times L_{i} + \beta X_{i,c,t} + \gamma_{c} + \alpha_{s} + \epsilon_{i,c,t}
\end{equation}

Here, \(L_{i}\) is a binary indicator equal to 1 if student \(i\) is in
the low-GPA group. The coefficient of interest is \(\mu\), which
captures the differential impact of the pandemic on low-performing
students relative to their high-performing peers. A negative \(\mu\)
would indicate that the pandemic exacerbated inequality in learning
outcomes. This specification mirrors the baseline model but excludes
continuous cumulative GPA from the vector \(X_{i,c,t}\), as it is
captured by the group classification.

\subsection{Identification of the Effect of Sudden Transition to Remote
Learning}\label{identification-of-the-effect-of-sudden-transition-to-remote-learning}

Finally, I isolate the effect of the sudden, forced transition to remote
learning. While the aggregate pandemic effect (\(\delta\) in Equation 1)
captures broadly defined disruptions, a key mechanism was the shift in
instructional modality. Prior to the pandemic, the department offered
the course in two distinct modes: Hybrid (one in-person and one online
session weekly) and Online (fully remote). The onset of the pandemic in
March 2020 forced all hybrid sections to transition abruptly to a fully
remote format. I exploit this variation using a
difference-in-differences framework to estimate the impact of
transitioning from hybrid to remote learning, relative to students who
were already enrolled in fully online sections:

\begin{equation}
y_{i,c,t} = \delta P_{t} + \phi O_{i} + \mu P_{t} \times O_{i} + \beta X_{i,c,t} + \gamma_{c} + \alpha_{s} + \epsilon_{i,c,t}
\end{equation}

In this specification, \(O_{i}\) is an indicator equal to 1 if the
student originally enrolled in an online section and 0 if they enrolled
in a hybrid section. The interaction term \(\mu\) tests whether students
who selected into online learning---and thus experienced less disruption
in modality---performed differently during the pandemic compared to
those forced to switch from hybrid to remote instruction. Control
variables remain consistent with the baseline specification, excluding
the instruction mode indicator which is subsumed by the
difference-in-differences terms.

\section{Results}\label{results}

\subsection{Average Course GPA Across Semesters in ECO
1001}\label{average-course-gpa-across-semesters-in-eco-1001}

\begin{figure}[H]

{\centering \pandocbounded{\includegraphics[keepaspectratio]{working_paper_2_files/figure-pdf/gpa-across-semesters-eco-1001-1.pdf}}

}

\caption{Average Course GPA in ECO 1001 across semesters}

\end{figure}%

An important argument I make in this paper is that student performance
is mostly measured using course completion, withdrawal rates, or GPA in
the literature. These may not be good measures of academic performance
during the pandemic. A sudden change in the educational setting also
affected instructors, who might have become more lenient with grading.
This change could have led to common exams being held online, giving
students more opportunities for possible misconduct. The possible
negative impact of the pandemic on students' actual performance could be
overshadowed by these changes in institutional policies and educational
settings. Using course GPA as a measure of student performance may
contradict students' experiences.

Figure 1 shows the unadjusted average GPA in the course ECO 1001 changes
over time. There is an abrupt jump in course GPA in Spring 2020, when
classes moved online in response to the pandemic. Studies based on
surveys of student experiences find that students faced substantial
hardships and struggled in their studies
(\citeproc{ref-aucejo_impact_2020}{Aucejo et al. 2020};
\citeproc{ref-rodriguez-planas_hitting_2020}{Rodríguez-Planas 2020}).
This would suggest a lower GPA in Spring 2020 --- if GPAs were a valid
measure of performance --- since students were less able to perform at
their best. Although GPAs decreased in Fall 2020 and Spring 2021, they
did not return to pre‑pandemic levels until after Fall 2021.

\subsection{Withdrawal Rate Across Semesters in ECO
1001}\label{withdrawal-rate-across-semesters-in-eco-1001}

\begin{figure}[H]

{\centering \pandocbounded{\includegraphics[keepaspectratio]{working_paper_2_files/figure-pdf/withdrawal-rate-1.pdf}}

}

\caption{Withdrawal rates in ECO 1001 across semesters}

\end{figure}%

Another mechanism that could explain the observed change in measured
performance is that the institution under study, like many others
nationwide, adopted a flexible grading policy to help students cope with
pandemic-related challenges. The policy attempted to reduce the burden
on students by offering three options through the last day of the
semester. The first option, Credit (CR), allowed students to receive
credit for the course while the grade did not affect their GPA. The
second option, No Credit (NC), allowed students to complete the course
without receiving credit and to retake it later without a record of
withdrawal. The third option was the standard course withdrawal.

Figure 2 shows unadjusted withdrawal-related rates for ECO 1001 across
semesters. In Spring 2020, the semester most affected by the pandemic
onset, the standard withdrawal rate was unusually low (3.92\%),
following the college's adoption of the flexible grading policy.
According to the figure, 29.75\% of students enrolled in ECO 1001 chose
CR and 5.53\% chose NC; only 3.92\% opted for a standard withdrawal in
Spring 2020. After Spring 2020, this flexible grading policy was not
renewed. The standard withdrawal rate rose to 6.65\% in Fall 2020 and
reached about 8\% by Spring 2022. Because the policy changed how courses
contributed to students' GPAs and completion records, using course GPA
or course completion rates may not accurately reflect the pandemic's
effect on academic performance.

\subsection{Descriptive Statistics}\label{descriptive-statistics}

\begin{verbatim}
[1] " "                               "Pre-Covid (N=752) / Mean"       
[3] "Pre-Covid (N=752) / Std. Dev."   "Post-Covid (N=3846) / Mean"     
[5] "Post-Covid (N=3846) / Std. Dev." "Diff. in Means"                 
[7] "Std. Error"                     
\end{verbatim}

\begin{table}[H]
\centering
\caption{\label{tab:descriptive-stats}Descriptive Statistics}
\centering
\resizebox{\ifdim\width>\linewidth\linewidth\else\width\fi}{!}{
\fontsize{9}{11}\selectfont
\begin{threeparttable}
\begin{tabular}[t]{lllll}
\toprule
\multicolumn{1}{c}{\em{\textbf{ }}} & \multicolumn{1}{c}{\em{\textbf{Pre-COVID (N = 752)}}} & \multicolumn{1}{c}{\em{\textbf{Post-COVID (N = 3846)}}} & \multicolumn{2}{c}{\em{\textbf{ }}} \\
\cmidrule(l{3pt}r{3pt}){2-2} \cmidrule(l{3pt}r{3pt}){3-3}
  & Mean & Mean & Diff. in Means & Std.error\\
\midrule
Final exam score & 56.586 & 57.148 & 0.562 & 0.606\\
Correct & 0.620 & 0.582 & -0.038 & 0.010\\
Hispanic & 0.133 & 0.189 & 0.056 & 0.014\\
Black & 0.082 & 0.077 & -0.005 & 0.011\\
Asian & 0.512 & 0.457 & -0.055 & 0.020\\
\addlinespace
Other race & 0.012 & 0.060 & 0.048 & 0.006\\
Fall & 0.480 & 0.542 & 0.062 & 0.020\\
Online & 0.346 & 0.585 & 0.239 & 0.019\\
GPA & 3.146 & 3.302 & 0.155 & 0.035\\
Female & 0.440 & 0.464 & 0.024 & 0.020\\
\addlinespace
Age & 21.352 & 20.219 & -1.133 & 0.178\\
Parttime & 0.082 & 0.051 & -0.031 & 0.011\\
Sophomore or below & 0.840 & 0.935 & 0.094 & 0.014\\
\bottomrule
\end{tabular}
\begin{tablenotes}[para]
\item \textit{Note: } 
\item Final exam scores are based on a 100-point scale.
\end{tablenotes}
\end{threeparttable}}
\end{table}

Table 1 outlines the sample characteristics before and after the
pandemic. The sample includes 4,598 students enrolled in the course. The
pre-COVID period covers observations from Spring and Fall 2019. The
post-COVID data includes students enrolled in Fall 2020, Spring 2021,
Fall 2021, and Spring 2022. The table reports the pre-Covid and
post-Covid averages of the variables as well as differences in their
means with standard errors.

On average, unadjusted difference in exam scores of the students in the
common final exams is 0.562 points. This difference is not statistically
significant. In case of performance on nearly identical questions, the
average probability of answering the question, unadjusted, is about 7.6
percentage points less in post-Covid exams relative to pre-Covid exams.
The difference is statistically significant at 1\% level. Regarding
student demographics, there has been an increase in the proportion of
Hispanic students in the course from 13.3 percent before the pandemic to
18.9 percent after. The enrollment proportion for Asian students has
decreased, with a difference of -5.5 percent. The proportion of Black
students has remained roughly the same before and after the pandemic,
with the small difference not being statistically significant. The
difference in enrollment for students of all races except for Black are
statistically significant at the 1 percent level. Before the pandemic,
around 35 percent of the students were enrolled in fully online classes.
However, in the post-Covid period, about 59 percent of students chose
fully online classes over hybrid classes. Notably, all students enrolled
in this course took fully remote classes during the Fall 2020 and Spring
2021 sessions. In contrast, during Fall 2021, all students were enrolled
in hybrid classes for the course. By Spring 2022, both hybrid and online
classes were available.

In the post-pandemic period, students are nearly a year younger than in
the pre-pandemic period, a difference that is statistically significant
at the 1 percent level. The proportion of part-time students has
decreased since 2019. The proportion of students whose native language
is not English has also decreased significantly from 58.1 percent to
43.1 percent. Most students taking the introductory microeconomics
course are freshmen or sophomores. Their proportion has increased by 10
percentage points in the post-pandemic period compared to the
pre-pandemic period. A crucial control variable in this study is the
students' GPA, for which I use their cumulative GPA from before the
semester in which they enrolled in the course started. Some observations
have missing values. If a student's cumulative GPA at the start of the
semester is missing, I replace it with their GPA at the end of the
semester. If a student's cumulative GPA before or after the semester is
missing, I impute the value using the mean GPA of the semester in which
the student enrolled in the course for further analyses.

\subsection{Baseline Specification}\label{baseline-specification}

\begin{table}[H]
\centering
\caption{\label{tab:base-specification}Baseline estimates of effects of COVID-19 on student performance}
\centering
\resizebox{\ifdim\width>\linewidth\linewidth\else\width\fi}{!}{
\fontsize{9}{11}\selectfont
\begin{threeparttable}
\begin{tabular}[t]{lcccc}
\toprule
\multicolumn{1}{c}{\em{\textbf{ }}} & \multicolumn{2}{c}{\em{\textbf{\makecell[c]{Final Exam Score\\(mean = 57.1, sd = 15.6)}}}} & \multicolumn{2}{c}{\em{\textbf{\makecell[c]{Did Student Get The Answer Correct (Y/N)?\\(mean = 0.6, sd = 0.49)}}}} \\
\cmidrule(l{3pt}r{3pt}){2-3} \cmidrule(l{3pt}r{3pt}){4-5}
  & (1) & (2) & (3) & (4)\\
\midrule
postcovid & -1.151 &  & -0.019*** & \\
 & (0.744) &  & (0.007) & \\
Fall 2020 &  & 1.308 &  & -0.102***\\
 &  & (1.504) &  & (0.013)\\
Spring 2021 &  & -5.760*** &  & -0.082***\\
\addlinespace
 &  & (1.135) &  & (0.015)\\
Fall 2021 &  & 5.643*** &  & -0.019**\\
 &  & (1.169) &  & (0.010)\\
Spring 2022 &  & -6.954*** &  & -0.088***\\
 &  & (1.370) &  & (0.012)\\
\addlinespace
Num.Obs. & 4598 & 4598 & 47589 & 47589\\
R2 & 0.209 & 0.223 & 0.036 & 0.039\\
\bottomrule
\end{tabular}
\begin{tablenotes}[para]
\item \textit{Note: } 
\item * p < 0.1, ** p < 0.05, *** p < 0.01. Final exam scores are based on a 100-point scale. Heteroskedasticity-robust standard errors are used. All regressions include the following control variables: cumulative GPA, gender, race, age, whether a student is at least a sophomore, part-time status of the student. All regressions also include a dummy variable, gpamiss, which is 1 if cumulative GPA is imputed using the mean and 0 otherwise. All regressions include course instructor fixed-effects and session fixed-effects.
\end{tablenotes}
\end{threeparttable}}
\end{table}

Table 3.2 presents the results of the baseline specification. As stated
earlier, student performance was measured using two outcome variables.
The coefficients with standard errors are reported. Also reported below
the standard errors are standardized coefficients in brackets. In the
first two columns, the outcome variable is the student's exam score on
the common final exam. It appears from a simple model in the first
column that performance measured using the exam score, decreased in the
post-pandemic period. Looking at the first column, on average, in the
post-pandemic period, the exam score decreased by a point (or 0.02
standard deviations), although the coefficient is not statistically
significant. Column 2 shows the results by semester using dummy
variables, with spring and Fall 2019 combined as the benchmark category.
Due to limited pre-pandemic observations, I combined spring and Fall
2019 data into a single period. When the pandemic struck, the score
increased by 1.35 points in Fall 2020 compared to exam scores in 2019,
but the coefficient is not statistically significant. The scores
decreased sharply in Spring 2021 by 5.75 points or 0.37 standard
deviations below the mean score. Mean scores increased in Fall 2021
before decreasing in Spring 2022 by 6.7 points (or 0.43 standard
deviations).

Columns 3-4 present the results from linear probability models, where
the outcome variable is binary since I look at the students' performance
on matched questions from pre and post pandemic final exam. For a full
period post pandemic, the probability of students answering a similar
question from pre pandemic exam decreases by 1.5 percentage points.
Analyzing the results across semesters, immediately after the pandemic
struck, I see a sharp decrease in the probability of students answering
a nearly identical question correctly in Fall 2020 compared to the
common final exams in 2019. The probability of answering the nearly
identical question in Fall 2020 decreased by 10 percentage points or
0.21 standard deviations below the mean probability compared to that of
in 2019. The performance appeared to improve in subsequent semesters,
with the probability of answering the nearly identical question from
pre-pandemic common exams during the pandemic decreased by about 8
percentage points in Spring 2022. In all regressions, I control for
students' demographic characteristics, including race and gender, as
well as other factors such cumulative GPA and their part-time student
status. In addition to that, in all regressions, I control for the
gpamiss variable to see if the results change due to mean imputation of
missing GPA values. The results do not appear to change due to that.

I also examine closely the differential effect of the pandemic based on
students' GPA quartiles for both exam scores and matched questions data.
For the exam scores dataset, the GPA quartiles are constructed as
follows: first quartile: GPA ≤ 3.01, second quartile: 3.01 \textless{}
GPA ≤ 3.37, third quartile: 3.37 \textless{} GPA ≤ 3.71, and fourth
quartile: GPA \textgreater{} 3.71. For matched questions data, the GPA
quartiles are constructed as follows: first quartile: GPA ≤ 3.08, second
quartile: 3.08 \textless{} GPA ≤ 3.32, third quartile: 3.32 \textless{}
GPA ≤ 3.68, and fourth quartile: GPA \textgreater{} 3.68. Using students
in the fourth GPA quartile as a benchmark group, I can examine how the
pandemic affected students in other quartiles.

\begin{table}[H]
\centering
\caption{\label{tab:gpa-by-quartiles}Differential effects of COVID-19 across GPA quartiles}
\centering
\resizebox{\ifdim\width>\linewidth\linewidth\else\width\fi}{!}{
\fontsize{9}{11}\selectfont
\begin{threeparttable}
\begin{tabular}[t]{lcccc}
\toprule
\multicolumn{1}{c}{\em{\textbf{ }}} & \multicolumn{2}{c}{\em{\textbf{\makecell[c]{Final Exam Score\\(mean = 57.1, sd = 15.6)}}}} & \multicolumn{2}{c}{\em{\textbf{\makecell[c]{Did Student Get The Answer Correct (Y/N)?\\(mean = 0.6, sd = 0.49)}}}} \\
\cmidrule(l{3pt}r{3pt}){2-3} \cmidrule(l{3pt}r{3pt}){4-5}
  & (1) & (2) & (3) & (4)\\
\midrule
postcovid & -1.894** & 1.564 & -0.022*** & 0.002\\
 & (0.739) & (1.421) & (0.007) & (0.010)\\
GPA (first quartile) & -18.335*** & -13.299*** & -0.202*** & -0.176***\\
 & (0.610) & (1.673) & (0.006) & (0.010)\\
GPA (second quartile) & -15.034*** & -11.586*** & -0.146*** & -0.116***\\
\addlinespace
 & (0.608) & (1.566) & (0.007) & (0.011)\\
GPA (third quartile) & -10.191*** & -8.107*** & -0.112*** & -0.119***\\
 & (0.573) & (1.958) & (0.006) & (0.011)\\
post x GPA (first quartile) &  & -5.722*** &  & -0.041***\\
 &  & (1.784) &  & (0.013)\\
\addlinespace
post x GPA (second quartile) &  & -3.946** &  & -0.059***\\
 &  & (1.699) &  & (0.014)\\
post x GPA (third quartile) &  & -2.362 &  & 0.008\\
 &  & (2.048) &  & (0.013)\\
Num.Obs. & 4598 & 4598 & 47589 & 47589\\
\addlinespace
R2 & 0.246 & 0.248 & 0.040 & 0.040\\
\bottomrule
\end{tabular}
\begin{tablenotes}[para]
\item \textit{Note: } 
\item * p < 0.1, ** p < 0.05, *** p < 0.01. Final exam scores are based on a 100-point scale. Heteroskedasticity-robust standard errors are used. All regressions include the following control variables: cumulative GPA, gender, race, age, whether a student is at most a sophomore, part-time status of the student. All regressions also include a dummy variable, gpamiss, which is 1 if cumulative GPA is imputed using the mean and 0 otherwise. All regressions include course instructor fixed-effects and session fixed-effects.
\end{tablenotes}
\end{threeparttable}}
\end{table}

In table 3.3, looking at the results from exam scores data, students in
the bottom quartile scored just over 18 points or 0.61 standard
deviations lower than students in the top quartile of the GPA
distribution. After the pandemic, this gap widened by 5.7 points (0.06
standard deviations) in the final exam scores. The gap in scores between
students in the third and top GPA quartiles widens by 2.36 points,
though this change is not statistically significant. Looking at the
results from matched-questions data, I observe a similar pattern. On
average students in bottom quartile are 20 percentage points less likely
to answer a nearly identical question compared to the students in top
quartile. In the post-pandemic period, the gap in mean probability of
answering a nearly identical question on a common exam widened by 4
percentage points or 0.02 standard deviations between students in the
top and bottom quartiles of the GPA distribution. These findings
demonstrate that students with lower GPAs experienced significantly
greater learning losses.

\subsection{Impact of COVID on Low GPA
Students}\label{impact-of-covid-on-low-gpa-students}

Panel A in table 3.4 show the results of differential impact of the
pandemic on the performance of students with low GPA compared to their
high GPA counterparts. As explained earlier, I define low GPA students
with GPA less than median GPA of 3.2. Columns 1 and 2 show results from
OLS regressions with final exam scores as the outcome variable. On
average, low GPA students score 11.4 points lower than high GPA students
on the common final exam. In column 2, I include an interaction term
that combines the low GPA dummy with a dummy for the post-COVID period.
This is similar to a standard difference-in-difference estimate of the
pandemic's effect on the performance of low GPA students relative to
high GPA students, where I assume the pandemic did not affect high GPA
students' performance. I see that due to the pandemic, the average exam
scores of low GPA students decreased by 3.3 points (or 0.04 standard
deviation) relative to high GPA students.

Comparing these results to those from linear probability models in
columns 3-4, I see a statistically significant reduction in the
performance of low GPA students. This is measured by their ability to
answer nearly identical questions in exams post-pandemic from the
pre-pandemic common exams. In column 3, I see that, on average, low GPA
students are 13.2 percentage points (or 0.13 standard deviations) less
likely to answer a similar question compared to their high GPA
counterparts. In column 4, the coefficient on an added interaction term
suggests that post-pandemic, low GPA students are 3.3 percentage points
less likely to answer a similar question from pre-pandemic common exams
compared to high GPA students. The coefficient is statistically
significant at the 1\% level.

\begin{table}[H]
\centering\begingroup\fontsize{9}{11}\selectfont

\resizebox{\ifdim\width>\linewidth\linewidth\else\width\fi}{!}{
\begin{threeparttable}
\begin{tabular}[t]{>{\raggedright\arraybackslash}p{15em}cccc}
\toprule
\multicolumn{1}{c}{\em{\textbf{ }}} & \multicolumn{2}{c}{\em{\textbf{\makecell[c]{Final Exam Score\\(mean = 57.1, sd = 15.6)}}}} & \multicolumn{2}{c}{\em{\textbf{\makecell[c]{Did Student Get The Answer Correct (Y/N)?\\(mean = 0.6, sd = 0.49)}}}} \\
\cmidrule(l{3pt}r{3pt}){2-3} \cmidrule(l{3pt}r{3pt}){4-5}
  & (1) & (2) & (3) & (4)\\
\midrule
postcovid & -2.637*** & -0.477 & -0.014* & 0.001\\
 & (0.750) & (1.139) & (0.007) & (0.008)\\
lowgpa & -11.369*** & -8.541*** & -0.132*** & -0.113***\\
 & (0.442) & (1.225) & (0.005) & (0.008)\\
post x lowgpa &  & -3.245** &  & -0.033***\\
\addlinespace
 &  & (1.303) &  & (0.010)\\
Num.Obs. & 4598 & 4598 & 47589 & \vphantom{1} 47589\\
R2 & 0.187 & 0.188 & 0.034 & 0.035\\
postcovid & -1.151 & 0.971 & -0.019*** & -0.040***\\
 & (0.744) & (0.962) & (0.007) & (0.009)\\
\addlinespace
online & -1.778*** & 2.987** & -0.076*** & -0.106***\\
 & (0.592) & (1.408) & (0.007) & (0.010)\\
post x online &  & -5.262*** &  & 0.060***\\
 &  & (1.428) &  & (0.014)\\
Num.Obs. & 4598 & 4598 & 47589 & 47589\\
\addlinespace
R2 & 0.209 & 0.211 & 0.036 & 0.037\\
\bottomrule
\end{tabular}
\begin{tablenotes}[para]
\item \textit{Note: } 
\item * p < 0.1, ** p < 0.05, *** p < 0.01. Heteroskedasticity-robust standard errors are used. All regressions include the following control variables: cumulative GPA, gender, race, and part-time status of the student. All regressions also include a dummy variable, gpamiss, which is 1 if cumulative
    GPA is imputed using the mean and 0 otherwise. All regressions include session fixed-effects and course instructor fixed-effects.
\end{tablenotes}
\end{threeparttable}}
\endgroup{}
\end{table}

\subsection{Abrupt Transition to Remote
Learning}\label{abrupt-transition-to-remote-learning}

Panel B in table 3.4 displays the results of the impact of the
pandemic-induced abrupt transition to remote learning from the
pre-pandemic hybrid mode of learning. Columns 1-2 present the results of
OLS models where the outcome variable is the final exam scores of the
students. On average, students enrolled in online classes score about
1.9 points less than those in hybrid classes, controlling for the COVID
period. In column 2, I interact a dummy variable for the COVID period
with a dummy variable for remote learning. The coefficient on the
interaction term can be interpreted as a difference-in-differences
estimate of the effect of suddenly transitioning from hybrid to online
classes. For the \emph{introductory microeconomics} course, pre-COVID,
the department offered both hybrid and online classes. When the pandemic
hit, the department followed the nationwide policy of abruptly
transitioning to online classes. The coefficient on the interaction term
thus presents the impact of this sudden shift to online learning from
hybrid learning on students' performance. The estimate is -5.432 (or
-0.06 standard deviations) and is statistically significant at the 1\%
level.

Columns 3-4 show the results of linear probability models with binary
outcome variable which is 1 if a student answers the question correctly
and 0 otherwise. Column 3 shows that on average, accounting for dummy
variable for the pandemic, students enrolled in online course are 7.3
percentage points less likely to answer a nearly identical question from
common exams from pre pandemic period in post pandemic exams. Column 4
is a classic difference-in-differences specification. Surprisingly, the
impact of a sudden transition from hybrid to online learning increased
the students' probability of answering a similar question from
pre-pandemic common exams in the post-pandemic period by 5.6 percentage
points.

In table 3.4, in panel A, all regressions include the following control
variables: instruction mode, gender, race, and part-time status of the
student. In panel B, all regressions include the following control
variables: cumulative GPA, gender, race, and part-time status of the
student. All regressions also include a dummy variable, gpamiss, which
is 1 if cumulative GPA is imputed using the mean and 0 otherwise. All
regressions include session fixed-effects and course instructor
fixed-effects.

\subsection{Dynamic Effects}\label{dynamic-effects}

I am also interested in examining the differential impact of the
pandemic on the outcomes of high and low GPA students across the
semesters. I interact the low GPA with separate time dummies for all
semesters, with Spring 2019 and Fall 2019 combined as the benchmark
category. This allows me to explore how the outcome differences between
low GPA and high GPA students evolve over time. I also perform the same
exercise to explore the impact of abrupt transition to online mode of
learning across the semesters. I interact a dummy variable for the
online mode of learning with all semester dummies, with the same
benchmark category.

In Figure 3.3, the outcome variable is scores on the common final exam.
The top-left panel illustrates the long-term impact of COVID-19 on exam
scores of low-GPA students compared to high-GPA students. There's a
sharp decline in exam scores for low-GPA students in Fall 2020. Although
their performance improves over time, a gap persists. The top-right
panel illustrates the impact of the sudden transition to online learning
on exam scores across different semesters. Immediately after the
COVID-19 hit, transition to online classes decreased exam scores but
recovered after one semester suggesting gradual adaptation to new
learning environment.

\begin{figure}[H]

{\centering \includegraphics[width=0.9\linewidth,height=\textheight,keepaspectratio]{working_paper_2_files/figure-pdf/dynamic-effects-1.pdf}

}

\caption{Dynamic effects of COVID-19 on students' performance}

\end{figure}%

A similar pattern emerges with matched question data used to measure
students' academic outcomes. The mean probability of answering a nearly
identical question post-pandemic compared to pre-pandemic exam decreases
sharply for low-GPA students immediately after COVID-19 hit (bottom-left
panel). It did not appear to recover by Spring 2022. The average
probability of answering a similar question correctly decreases due to
the sudden transition to online classes but then increases to the levels
seen before the pandemic (bottom-right panel).

\subsection{Effects Due to Heterogenity in Difficulty of
Questions}\label{effects-due-to-heterogenity-in-difficulty-of-questions}

This section assesses the effect of the COVID‑19 pandemic on student
performance by comparing the mean probability of correctly answering
matched (nearly identical) exam items administered before and during the
pandemic, conditioning on item difficulty. All ECO 1001 students take a
common final exam; instructors pre-classified exam items as ``easy'' or
``hard.'' In the matched‑item analysis I preserve those pre‑pandemic
difficulty labels---an item is treated as ``hard'' (or ``easy'') if it
was so designated in the pre‑pandemic exams---thereby isolating changes
in student performance from contemporaneous shifts in exam composition.

\begin{table}[H]
\centering\begingroup\fontsize{9}{11}\selectfont

\resizebox{\ifdim\width>\linewidth\linewidth\else\width\fi}{!}{
\begin{threeparttable}
\begin{tabular}[t]{>{\raggedright\arraybackslash}p{15em}cccc}
\toprule
\multicolumn{1}{c}{\em{\textbf{ }}} & \multicolumn{2}{c}{\em{\textbf{\makecell[c]{Easy Questions \\ (mean = 0.645, sd = 0.479)}}}} & \multicolumn{2}{c}{\em{\textbf{\makecell[c]{Hard Questions \\ (mean = 0.573, sd = 0.495)}}}} \\
\cmidrule(l{3pt}r{3pt}){2-3} \cmidrule(l{3pt}r{3pt}){4-5}
  & (1) & (2) & (3) & (4)\\
\midrule
postcovid & -0.006 & -0.018 & -0.003 & 0.014\\
 & (0.018) & (0.020) & (0.010) & (0.011)\\
lowgpa & -0.092*** & -0.101*** & -0.141*** & -0.120***\\
 & (0.009) & (0.013) & (0.006) & (0.010)\\
post x lowgpa &  & 0.022 &  & -0.037***\\
\addlinespace
 &  & (0.019) &  & (0.013)\\
Num.Obs. & 13332 & 13332 & 28018 & \vphantom{1} 28018\\
R2 & 0.033 & 0.033 & 0.037 & 0.037\\
postcovid & 0.003 & -0.005 & -0.009 & -0.044***\\
 & (0.018) & (0.021) & (0.010) & (0.012)\\
\addlinespace
online & -0.126*** & -0.135*** & -0.047*** & -0.093***\\
 & (0.017) & (0.022) & (0.010) & (0.013)\\
post x online &  & 0.019 &  & 0.100***\\
 &  & (0.027) &  & (0.018)\\
Num.Obs. & 13332 & 13332 & 28018 & 28018\\
\addlinespace
R2 & 0.031 & 0.031 & 0.042 & 0.043\\
\bottomrule
\end{tabular}
\begin{tablenotes}[para]
\item \textit{Note: } 
\item * p < 0.1, ** p < 0.05, *** p < 0.01. Heteroskedasticity-robust standard errors are used. All regressions include the following control variables: cumulative GPA, gender, race, and part-time status of the student. All regressions also include a dummy variable, gpamiss, which is 1 if cumulative
    GPA is imputed using the mean and 0 otherwise. All regressions include session fixed-effects and course instructor fixed-effects.
\end{tablenotes}
\end{threeparttable}}
\endgroup{}
\end{table}

Results from panel A in table 3.5 show that on average low GPA students
are about 15 percentage points (0.15 standard deviations) less likely to
answer a hard question correctly compared to high GPA students. Due to
the pandemic, low-GPA students' mean probability of answering nearly
identical hard questions correctly decreased by 4.5 percentage points
(or 0.02 standard deviations) compared to high-GPA students. For easy
questions, their performance decreased by 1.6 percentage points, though
this estimate is not statistically significant. In Panel B, I do a
similar analysis for students enrolled in online relative to hybrid
classes. On average, students enrolled in online classes are 7.2
percentage points (or 0.07 standard deviations) less likely to answer a
hard question correctly compared to students enrolled in hybrid classes.
Following the abrupt transition from pre-pandemic hybrid learning to
online mode, mean probability of answering nearly identical hard
questions correctly increased by just over 12 percentage points (0.06
standard deviations). For easy questions, the estimate is not
statistically significant.

\section{Conclusion}\label{conclusion}

In this essay, I examine the pandemic's influence on the academic
performance of students by analyzing their results in the common exams
for introductory microeconomics course at a large public university in
New York City. I advance the literature by providing estimates of
learning loss in college students due to pandemic that are more reliable
than current estimates. I use two outcome measures to evaluate students'
academic performance and argue that these outcome choices are more
appropriate than the existing outcome measures such as course completion
rate, course GPA, or semester GPA used in the literature on the impact
of COVID on students' academic performance. First, I analyze students'
scores on common final exams administered at the institution from 2019
to 2022, excluding Spring 2020 due to lack of data availability for that
semester. Acknowledging the fact that difficulty of exams may have
changed during the pandemic, I use 35 pairs of questions matched from
these common final exams to measure changes in the students' average
probability of answering nearly identical questions from the exams
conducted before and during the pandemic to eliminate the variation from
exam difficulty. I find an overall negative impact of the pandemic on
students' outcomes. Students' scores went down by a point (or 0.02
standard deviations) in the full pandemic period (2020-2022), although
the coefficient is not statistically significant. Students' average
probability of answering similar questions from the common exams before
the pandemic went down during the pandemic by 1.5 percentage points.
This clear evidence of learning loss, I argue, is not affected by the
flexible grading policy. The extent of learning loss was greater in Fall
2020 and gradually lessened through Fall 2021, after which it
stabilized.

I also examine the differential impact of the pandemic on the outcomes
of students with low GPA compared to those with high GPA. My findings
suggest that on average low GPA students have a 3.3 percentage point
lower average probability of correctly answering similar questions
compared to high GPA students during the pandemic. This accounts for a
broad range of student characteristics and incorporates instructor and
session fixed effects, indicating a significant differential impact on
low GPA students. While using students' scores from common exams as the
outcome variable, I find that low GPA students on average scored 3.23
points (or 0.04 standard deviations) less in the common exams compared
to high GPA students during the pandemic. In the long term, although
this difference decreases, it does not return to the pre-pandemic level
by Spring 2022. Additionally, I examined the pandemic's effects across
GPA quartiles and found that students in the lowest quartile of GPA
distribution were 4.1 percentage points (0.02 standard deviations) less
likely to correctly answer nearly identical questions from pre-pandemic
exams during the pandemic. This analysis supports the hypothesis that
low GPA students, on average, suffered greater learning loss due to the
pandemic compared to high GPA students.

Furthermore, I explore an important channel: the sudden shift to online
classes, through which the pandemic affected students' academic
outcomes. I find that abruptly moving to online classes due to the
pandemic reduced students' final exam scores by 5.43 points. In case of
matched questions data, the mean probability of answering a similar
question before and after suddenly moving to online classes increased by
5.6 percentage points. Interacting the semester dummies with a dummy for
online variable, I find that the abrupt transition to online classes
reduced the average probability of answering a similar question
correctly before and during pandemic before returning to pre-pandemic
levels. The same pattern is observed in case of exam scores as outcome
variable. To examine how sensitive these estimates of learning loss are
to question difficulty in the matched questions data, I provide results
from separate analyses using easy as well as hard questions. During the
pandemic, low-GPA students' mean probability of answering nearly
identical hard questions decreased by 4.5 percentage points relative to
their high-GPA counterparts. I found no statistically significant effect
for easy questions. When examining the effect of abrupt transition to
remote classes, I found that students scored just over 12 percentage
points higher on hard questions after moving online, while showing no
statistically significant difference on easy questions.

Overall, I find negative effects of the pandemic on students' academic
performance that align directionally with the current literature. My
unique matched questions data allows me to eliminate bias in the
estimates that arose from flexible grading policies implemented
immediately after the pandemic hit educational institutions nationwide.
I do, however, acknowledge that my estimates may not account fully for
potential cheating by students, especially in the initial months
following a sudden transition to remote classes. The implications of
learning loss due to the pandemic could be significant. On one hand,
students' GPAs, both course-specific and overall, did not change much or
even increased in some cases during the pandemic, giving the impression
of better performance. On the other hand, evidence from student surveys
shows that students faced hardships and challenges in learning during
this time. In my study I provide evidence of learning loss which is
consistent with students' negative experiences during the pandemic. In
future, any decision to suddenly switch to remote learning during a
complex situation should be carefully considered before implementation.

\pagebreak

\section{References}\label{references}

\phantomsection\label{refs}
\begin{CSLReferences}{1}{0}
\bibitem[\citeproctext]{ref-alpert_randomized_2016}
Alpert, William T., Kenneth A. Couch, and Oskar R. Harmon. 2016. {``A
{Randomized} {Assessment} of {Online} {Learning}.''} \emph{American
Economic Review} 106 (5): 378--82.
\url{https://doi.org/10.1257/aer.p20161057}.

\bibitem[\citeproctext]{ref-altindag_is_2021}
Altindag, Duha Tore, Elif S. Filiz, and Erdal Tekin. 2021. {``Is
{Online} {Education} {Working}?''} Working \{Paper\}. Working {Paper}
{Series}. National Bureau of Economic Research.
\url{https://doi.org/10.3386/w29113}.

\bibitem[\citeproctext]{ref-aucejo_impact_2020}
Aucejo, Esteban M., Jacob French, Maria Paola Ugalde Araya, and Basit
Zafar. 2020. {``The Impact of {COVID}-19 on Student Experiences and
Expectations: {Evidence} from a Survey.''} \emph{Journal of Public
Economics} 191 (November): 104271.
\url{https://doi.org/10.1016/j.jpubeco.2020.104271}.

\bibitem[\citeproctext]{ref-bettinger_virtual_2017}
Bettinger, Eric P., Lindsay Fox, Susanna Loeb, and Eric S. Taylor. 2017.
{``Virtual {Classrooms}: {How} {Online} {College} {Courses} {Affect}
{Student} {Success}.''} \emph{American Economic Review} 107 (9):
2855--75. \url{https://doi.org/10.1257/aer.20151193}.

\bibitem[\citeproctext]{ref-bird_negative_2022}
Bird, Kelli A., Benjamin L. Castleman, and Gabrielle Lohner. 2022.
{``Negative {Impacts} from the {Shift} to {Online} {Learning} During the
{COVID}-19 {Crisis}: {Evidence} from a {Statewide} {Community} {College}
{System}.''} \emph{AERA Open} 8 (1).
\url{https://doi.org/10.1177/23328584221081220}.

\bibitem[\citeproctext]{ref-bonacini_unraveling_2023}
Bonacini, Luca, Giovanni Gallo, and Fabrizio Patriarca. 2023.
{``Unraveling the Controversial Effect of {Covid}-19 on College
Students' Performance.''} \emph{Scientific Reports} 13 (1): 15912.
\url{https://doi.org/10.1038/s41598-023-42814-7}.

\bibitem[\citeproctext]{ref-cacault_distance_2021}
Cacault, M Paula, Christian Hildebrand, Jérémy Laurent-Lucchetti, and
Michele Pellizzari. 2021. {``Distance {Learning} in {Higher}
{Education}: {Evidence} from a {Randomized} {Experiment}.''}
\emph{Journal of the European Economic Association} 19 (4): 2322--72.
\url{https://doi.org/10.1093/jeea/jvaa060}.

\bibitem[\citeproctext]{ref-escueta_education_2017}
Escueta, Maya, Vincent Quan, Andre Joshua Nickow, and Philip Oreopoulos.
2017. {``Education {Technology}: {An} {Evidence}-{Based} {Review},''}
August, w23744. \url{https://doi.org/10.3386/w23744}.

\bibitem[\citeproctext]{ref-fuchs-schundeln_covid-induced_2022}
Fuchs-Schündeln, Nicola. 2022. {``Covid-{Induced} {School} {Closures} in
the {US} and {Germany}: {Long}-{Term} {Distributional} {Effects}.''}
\emph{CESifo Working Paper Series}.
\url{https://ideas.repec.org//p/ces/ceswps/_9698.html}.

\bibitem[\citeproctext]{ref-grewenig_covid-19_2021}
Grewenig, Elisabeth, Philipp Lergetporer, Katharina Werner, Ludger
Woessmann, and Larissa Zierow. 2021. {``{COVID}-19 and Educational
Inequality: {How} School Closures Affect Low- and High-Achieving
Students.''} \emph{European Economic Review} 140 (November): 103920.
\url{https://doi.org/10.1016/j.euroecorev.2021.103920}.

\bibitem[\citeproctext]{ref-ives_did_2024}
Ives, Bob, and Ana-Maria Cazan. 2024. {``Did the {COVID}-19 Pandemic
Lead to an Increase in Academic Misconduct in Higher Education?''}
\emph{Higher Education} 87 (1): 111--29.
\url{https://doi.org/10.1007/s10734-023-00996-z}.

\bibitem[\citeproctext]{ref-jaeger_global_2021}
Jaeger, David A., Jaime Arellano-Bover, Krzysztof Karbownik, Marta
Martínez Matute, John M. Nunley, Jr Seals, Miguel Almunia, et al. 2021.
{``The {Global} {COVID}-19 {Student} {Survey}: {First} {Wave}
{Results}.''} Working \{Paper\} 14419. IZA Discussion Papers.
\url{https://www.econstor.eu/handle/10419/236450}.

\bibitem[\citeproctext]{ref-jaggars_how_2016}
Jaggars, Shanna Smith, and Di Xu. 2016. {``How Do Online Course Design
Features Influence Student Performance?''} \emph{Computers \& Education}
95 (April): 270--84.
\url{https://doi.org/10.1016/j.compedu.2016.01.014}.

\bibitem[\citeproctext]{ref-jenkins_when_2023}
Jenkins, Baylee D., Jonathan M. Golding, Alexis M. Le Grand, Mary M.
Levi, and Andrea M. Pals. 2023. {``When {Opportunity} {Knocks}:
{College} {Students}' {Cheating} {Amid} the {COVID}-19 {Pandemic}.''}
\emph{Teaching of Psychology} 50 (4): 407--19.
\url{https://doi.org/10.1177/00986283211059067}.

\bibitem[\citeproctext]{ref-joyce_does_2015}
Joyce, Ted, Sean Crockett, David A. Jaeger, Onur Altindag, and Stephen
D. O'Connell. 2015. {``Does Classroom Time Matter?''} \emph{Economics of
Education Review} 46 (June): 64--77.
\url{https://doi.org/10.1016/j.econedurev.2015.02.007}.

\bibitem[\citeproctext]{ref-kofoed_zooming_2021}
Kofoed, Michael S., Lucas Gebhart, Dallas Gilmore, and Ryan Moschitto.
2021. {``Zooming to {Class}?: {Experimental} {Evidence} on {College}
{Students}' {Online} {Learning} During {COVID}-19.''}
\url{https://www.iza.org/publications/dp/14356/zooming-to-class-experimental-evidence-on-college-students-online-learning-during-covid-19}.

\bibitem[\citeproctext]{ref-rodriguez-planas_hitting_2020}
Rodríguez-Planas, Núria. 2020. {``Hitting {Where} It {Hurts} {Most}:
{Covid}-19 and {Low}-{Income} {Urban} {College} {Students}.''} \{SSRN\}
\{Scholarly\} \{Paper\}. Rochester, NY.
\url{https://doi.org/10.2139/ssrn.3682958}.

\bibitem[\citeproctext]{ref-rodriguez-planas_covid-19_2022}
---------. 2022. {``{COVID}-19, College Academic Performance, and the
Flexible Grading Policy: {A} Longitudinal Analysis.''} \emph{Journal of
Public Economics} 207 (March): 104606.
\url{https://doi.org/10.1016/j.jpubeco.2022.104606}.

\bibitem[\citeproctext]{ref-walsh_why_2021}
Walsh, Lisa L., Deborah A. Lichti, Christina M. Zambrano-Varghese,
Ashish D. Borgaonkar, Jaskirat S. Sodhi, Swapnil Moon, Emma R. Wester,
and Kristine L. Callis-Duehl. 2021. {``Why and How Science Students in
the {United} {States} Think Their Peers Cheat More Frequently Online:
Perspectives During the {COVID}-19 Pandemic.''} \emph{International
Journal for Educational Integrity} 17 (1): 1--18.
\url{https://doi.org/10.1007/s40979-021-00089-3}.

\bibitem[\citeproctext]{ref-xu_promises_2019}
Xu, Di, and Ying Xu. 2019. {``The {Promises} and {Limits} of {Online}
{Higher} {Education}: {Understanding} {How} {Distance} {Education}
{Affects} {Access}, {Cost}, and {Quality}.''} American Enterprise
Institute. \url{https://eric.ed.gov/?id=ED596296}.

\end{CSLReferences}

\pagebreak

\section{Appendix}\label{appendix}

\subsection{Average final exam scores in ECO 1001 across
semesters}\label{average-final-exam-scores-in-eco-1001-across-semesters}

\begin{figure}[H]

{\centering \pandocbounded{\includegraphics[keepaspectratio]{working_paper_2_files/figure-pdf/exam-scores-mean-1.pdf}}

}

\caption{Average Final Exam Scores in ECO 1001 across Semesters}

\end{figure}%

\subsection{Student shares in low and high GPA
groups}\label{student-shares-in-low-and-high-gpa-groups}

\begin{figure}[H]

{\centering \pandocbounded{\includegraphics[keepaspectratio]{working_paper_2_files/figure-pdf/share-gpa-low-high-1.pdf}}

}

\caption{Share of High vs Low GPA Students}

\end{figure}%

\subsection{Average GPA in low and high GPA groups in ECO 1001 across
semesters}\label{average-gpa-in-low-and-high-gpa-groups-in-eco-1001-across-semesters}

\begin{figure}[H]

{\centering \pandocbounded{\includegraphics[keepaspectratio]{working_paper_2_files/figure-pdf/meangpa-gpa-group-1.pdf}}

}

\caption{Average GPA in High vs Low GPA Group of Students}

\end{figure}%

\subsection{Share of Students in hybrid and online
classes}\label{share-of-students-in-hybrid-and-online-classes}

\begin{figure}[H]

{\centering \pandocbounded{\includegraphics[keepaspectratio]{working_paper_2_files/figure-pdf/student-share-online-hybrid-1.pdf}}

}

\caption{Share of the Students in Hybrid vs Online Classes}

\end{figure}%

\pagebreak

\subsection{Example of a
Matched-Question}\label{example-of-a-matched-question}

As explained earlier, I was able to match 35 pairs of nearly identical
questions from pre-pandemic common exams to exams conducted during the
pandemic. I provide an example of one such question below that was
similar in common final exams in Fall 2019 and Fall 2020 which was
deemed to be \emph{hard} by the instructors. Full list of matched
questions are provided in a separate document.

\subsubsection{Fall 2019 version}\label{fall-2019-version}

\noindent Scenario 2, Monopoly: Let the following equations the market
for energy for ConEd, a monopolist: \(P=56-2Q\), \(MR=56-4Q\),
\(TC=50+6Q+3Q^2\), \(MC=6+6Q\)

\noindent Refer to Scenario 2, Monopoly: What is the profit of ConEd at
the profit maximizing quantity? (round to the nearest whole number and
pick the best answer)

\begin{enumerate}
\def\labelenumi{\alph{enumi})}
\tightlist
\item
  100
\item
  50
\item
  75
\item
  155
\end{enumerate}

\subsubsection{Fall 2020 version}\label{fall-2020-version}

\noindent A monopolist has a total cost curve represented by
\(TC=50+2Q+Q^2\), and a marginal cost curve represented by \(MC=2+2Q\).
The monopolist faces the demand curve \(P=100-3Q\). The price is in
dollars and the quantity is in thousands. What is the monopolist's
profit? (pick the closest answer)

\begin{enumerate}
\def\labelenumi{\alph{enumi})}
\tightlist
\item
  \$330,330
\item
  \$550,250
\item
  \$750,000
\item
  \$1,000,600
\end{enumerate}




\end{document}
